\documentclass{article}
\usepackage{xeCJK}
\usepackage{xunicode,xltxtra}
\usepackage{everb}
\usepackage{verbatim}
\usepackage{natbib}
\usepackage[colorlinks, linkcolor=red, anchorcolor=purple,citecolor=blue]{hyperref}

% \cite{Liu:05} => Liu et al. (2005)
% \citep{Liu:05} => (Liu et al., 2005)

\everbsetup{bgcolor={[rgb]{0.92,0.92,1.00}}}

\def\argmax{\mathop{\rm argmax}}%
\def\argmin{\mathop{\rm argmin}}%
\def\max{\mathop{\rm max}}%
\renewcommand\refname{参考文献}

\title{THUMT: 开源神经机器翻译工具包}
\author{清华大学自然语言处理与人文计算实验室}
\date{2017年12月}

\begin{document}
\maketitle

\section{简介}
机器翻译(Machine Translation,MT)是使用计算机来自动进行人类语言翻译,它是自然语言处理以及人工智能的一个重要任务。自1990年以来,随着平行语料的增长,数据驱动的机器翻译方法变得流行起来。近年来,端到端的神经机器翻译(Neural Machine Translation,NMT)\citep{Sutskever:14,Bahdanau:15}进展迅速。由于NMT能够从数据中学习到特征表示,NMT迅速地取代了传统的统计机器翻译方法(Statistical Machine Translation,SMT)\citep{Brown:93,Koehn:03,Chiang:05},成为了实用MT系统中新的事实上的标准 \citep{Wu:16}。

基于\href{http://deeplearning.net/software/theano/)}{TensorFlow}, THUMT是一套开源的神经机器翻译工具包,由\href{http://nlp.csai.tsinghua.edu.cn/site2/index.php?lang=zh}{清华大学自然语言处理与人文实验室开发}。THUMT具有以下特性:

\begin{enumerate}
\item {\em 多神经机器翻译架构}。THUMT实现了传统的序列到序列架构\citep{Sutskever:14},标准的基于注意力机制的编码器-解码器架构\citep{Bahdanau:15}以及基于自注意力机制的Transfromer架构\citep{vaswani2017attention}。
\item {\em 最小错误率训练}。除了标准的最大似然估计(Maximum Likelihood Estimation,MLE), THUMT也支持最小错误率训练(Minimum Risk Training,MRT)\citep{Shen:16}。最小错误率训练旨在寻找能够最小化错误期望的参数,而错误期望通过BLEU\citep{Papineni:02}等评测目标在训练集上计算。
 \end{enumerate}

\section{安装}

\subsection{系统要求}
THUMT支持Linux i686以及MacOS X。必须安装以下第三方软件以支持THUMT:
\begin{enumerate}
\item Python版本2.7.0或以上。
\item JRE 1.6或以上(可选,仅当可视化时需要)。
\end{enumerate}

\subsection{安装前提}
我们推荐使用pip来安装THUMT的依赖。安装从python-pip开始:

\begin{everbatim}
apt-get install python-pip
\end{everbatim}

然后,使用以下两个命令来安装argparse以及TensorFlow(要求版本>=1.4.0):

\begin{everbatim}
pip install argparse
pip install tensorflow-gpu
\end{everbatim}

\subsection{安装THUMT}

THUMT的源代码可以从\href{http://thumt.thunlp.org}{工具网站}(稳定版本)以及GitHub(最新版本)中获得。以下是安装THUMT的简单指南。

\subsubsection{第一步: 解压}
使用以下命令进行解压:

\begin{everbatim}
tar xvfz THUMT.tar.gz
\end{everbatim}

进入\verb|THUMT|目录,可以找到两个子目录(\verb|thumt|, 和\verb|data|)以及三个文件(\verb|LICENSE|,\verb|UserManual.en.pdf|和\verb|UserManual.zh.pdf|):
\begin{enumerate}
\item \verb|thumt|: 源代码。
\item \verb|data|: 训练、验证以及测试集的玩具样例。
\item \verb|docs|: 文档源文件。
\item \verb|LICENSE|: 许可文件。
\item \verb|UserManual.en.pdf|:英文文档。
\item \verb|UserManual.zh.pdf|: 本文档。
\end{enumerate}


\subsubsection{第二步:更改环境变量}

我们极力推荐在GPU服务器上运行THUMT。假定THUMT在NVIDIA GPUs运行并且安装了\href{https://developer.nvidia.com/cuda-toolkit}{CUDA工具包}版本8.0,用户需要设置以下环境变量来使用THUMT以及GPU:
\begin{everbatim}
export PATH=/usr/local/cuda/bin:$PATH
export LD_LIBRARY_PATH=/usr/local/cuda/lib64:$LD_LIBRARY_PATH
export PYTHONPATH=/PATH/TO/THUMT:$PYTHONPATH
\end{everbatim}

若希望永久设置这些环境变量,用户可以将以上几行添加到\verb|$HOME|目录的\verb|.bashrc|文件中。其中\verb|/PATH/TO/THUMT|是THUMT所在目录。

\section{使用手册}

\subsection{数据准备}

运行THUMT会涉及到三种数据集:
\begin{enumerate}
\item {\em 训练集}: 用于训练NMT的平行句对集合。
\item {\em 验证集}: 由源端句子以及多个目标端翻译组成的数据集合,用于模型选择以及超参调节。
\item {\em 测试集}: 由源端句子以及多个目标端翻译组成的数据集合,用于测试模型在未知数据上的表现。
\end{enumerate}

\subsubsection{训练集}

训练集的作用是来训练NMT模型。它一般由两个文件组成:一个保存源端句子而另一个保存对应的目标端句子。在\verb|data|目录下,又一个名为\verb|train.src|的例示文件,它包含以下7句话:
\begin{everbatim}
我 很 喜欢 音乐 。

我 不 喜欢 画画 。

你 喜欢 音乐 么 ?

是的 , 我 也 喜欢 音乐 。

他 也 喜欢 音乐 。

她 一点 都 不 喜欢 音乐 。

她 很 喜欢 画画 。

\end{everbatim}

如下所示是对应的目标文件\verb|train.trg|的内容:
\begin{everbatim}
i like music very much .
i do not like painting .
do you like music ?
yes , i like music too .
he also likes music .
she does not like music at all .
she likes painting very much .
\end{everbatim}

注意源端和目标端文件中的每一行之包含一句分词或token后的句子。相同行号的源句子和目标句子互为翻译。

我们的玩具训练集只包含7个句子。在实践中,NMT通常需要数百万的平行句对才能获得较好的翻译结果。

\subsubsection{验证集}

验证集用来模型选择以及超参调整。在训练过程中,THUMT每隔一段时间在验证集上进行模型验证。在验证集上获得最高BLEU的模型将会被保存起来。

验证集包含一个源文件以及一个或多个目标文件。在\verb|data|目录下有一个例示源文件\verb|valid.src|,内容如下:

\begin{everbatim}
我 很 喜欢 画画 。

我 不 喜欢 音乐 。

\end{everbatim}

我们的例示验证集之包含一个目标文件\verb|valid.trg|,它包含文件\verb|valid.src|的翻译:
\begin{everbatim}
i like painting very much .
i do not like music .
\end{everbatim}

由于自然语言表达非常丰富,一个源端句子一般对应多个参考译文。THUMT也支持使用多个参考译文。

\subsubsection{测试集}
测试集用来评估训练好的NMT模型在未见数据上的表现。与验证集类似,测试集也包好一个源文件一个一个或多个目标文件。

在\verb|data|目录下,有一个源测试文件\verb|test.src|:
\begin{everbatim}
她 喜欢 画画 么 ?

他 一点 都 不 喜欢 画画 。

\end{everbatim}

对应的目标文件\verb|test.trg|内容如下:
\begin{everbatim}
does she like painting ?
he does not like painting at all .
\end{everbatim}

在我们的例子中,验证集和测试集只包含两个句子。在时间中,验证集和测试集通常包含数千个句子。

\subsection{训练}

\subsection{数据处理}
为了解决NMT所存在的词表限制问题,最常用的方式是使用Byte Pair Encoding (BPE)方法。BPE的源代码可以在\href{https://github.com/rsennrich/subword-nmt}{Subword-NMT}中找到。

为了使用BPE来编码训练集,首先需要生成BPE操作文件。以下命令将会生成一个名为\verb|bpe32k|的文件,它包含32K个BPE操作。该命令同时生成两个名为\verb|vocab.src|和\verb|vocab.trg|的文件。
\begin{everbatim}
python subword-nmt/learn_joint_bpe_and_vocab.py 
  --input train.src train.trg -s 32000 -o bpe32k 
  --write-vocabulary vocab.src vocab.trg
\end{everbatim}

在生成BPE操作文件以及词表以后,需要对训练集、验证集以及测试集源端进行编码:
\begin{everbatim}
  python subword-nmt/apply_bpe.py 
    --vocabulary vocab.src 
    --vocabulary-threshold 50 
    -c bpe32k < train.src > train.32k.src
  python subword-nmt/apply_bpe.py
    --vocabulary vocab.trg 
    --vocabulary-threshold 50
    -c bpe32k < train.trg > train.32k.trg
  python subword-nmt/apply_bpe.py
    --vocabulary vocab.src
    --vocabulary-threshold 50
    -c bpe32k < valid.src > valid.32k.src
  python subword-nmt/apply_bpe.py 
    --vocabulary vocab.trg --vocabulary-threshold 50 
    -c bpe32k < valid.trg > valid.32k.trg
  python subword-nmt/apply_bpe.py
    --vocabulary vocab.src 
    --vocabulary-threshold 50 
    -c bpe32k < test.src > test.32k.src
\end{everbatim}

以下命令可以将BPE编码后的文件进行还原:
\begin{everbatim}
sed -r 's/(@@ )|(@@ ?$)//g' < input > output
\end{everbatim}


\subsection{数据随机化}
将训练数据随机打乱顺序对训练模型很有帮助。为此我们提供了\verb|shuffle_corpus.py|脚本对数据进行打乱,它具有以下参数:
\begin{everbatim}
usage: shuffle_corpus.py [-h] --corpus CORPUS [CORPUS ...] 
                         [--suffix SUFFIX]
                         [--seed SEED]

Shuffle corpus

optional arguments:
  -h, --help            show this help message and exit
  --corpus CORPUS [CORPUS ...]
                        input corpora
  --suffix SUFFIX       Suffix of output files
  --seed SEED           Random seed
\end{everbatim}

各个参数的含义如下:
\begin{enumerate}
\item \verb|corpus|:源端以及目标端语料库。
\item \verb|--suffix|:输出文件的后缀,默认\verb|.shuf|。
\item \verb|--seed|:随机数种子。
\item \verb|--help|:显示帮助信息。
\end{enumerate}


\subsubsection{词表生成}
在训练之前,首先要根据语料库构建词表。\verb|scripts|下的\verb|build_vocab.py|可以完成这个功能。它具有以下参数:
\begin{everbatim}
usage: build_vocab.py [-h] [--limit LIMIT] [--control CONTROL]
                      corpus output

Create vocabulary

positional arguments:
  corpus             input corpus
  output             Output vocabulary name

optional arguments:
  -h, --help         show this help message and exit
  --limit LIMIT      Vocabulary size
  --control CONTROL  Add control symbols to vocabulary.
                     Control symbols are separated by comma.
\end{everbatim}

各个参数的含义如下:
\begin{enumerate}
\item \verb|corpus|:源端或目标端语料库。
\item \verb|--references|:输出词表名。
\item \verb|--limit|:词表限制大小。若不提供则使用语料库中所有出现的词。
\item \verb|--control|:控制字符,以逗号隔开如``\verb|<pad>,<eos>,<unk>|'',控制字符将添加到词表的最前端。
\item \verb|--help|:显示帮助信息。
\end{enumerate}

\subsubsection{训练所使用的Python脚本}
THUMT实现了传统的序列到序列架构Seq2Seq\citep{Sutskever:14},标准的基于注意力机制的编码器-解码器架构RNNsearch\citep{Bahdanau:15}以及基于自注意力机制的Transfromer架构\citep{vaswani2017attention}。

\verb|bin|目录下的\verb|trainer.py|脚本用于训练NMT模型。它有以下参数:
\begin{everbatim}
usage: trainer.py [<args>] [-h | --help]

Training neural machine translation models

optional arguments:
  -h, --help            show this help message and exit
  --input INPUT INPUT   Path of source and target corpus
  --output OUTPUT       Path to saved models
  --vocabulary VOCABULARY VOCABULARY
                        Path of source and target vocabulary
  --validation VALIDATION
                        Path of validation file
  --references REFERENCES [REFERENCES ...]
                        Path of reference files
  --model MODEL         Name of the model
  --parameters PARAMETERS
                        Additional hyper parameters
\end{everbatim}

我们区分\textit{必须}参数以及\textit{可选}参数。用户必须指定以下的参数来运行训练脚本\verb|trainer.py|:
\begin{enumerate}
\item \verb|--model|:所选择的架构,目前支持三种架构。可选择的值为\verb|seq2seq|,\verb|rnnsearch|以及\verb|transformer|。
\item \verb|--input|:首次运行时需要指定,表示源端与目标端训练文件。
\item \verb|--vocabulary|:首次运行时需要指定,表示源端与目标端词表所在路径。
\end{enumerate}

\verb|trainer.py|脚本中的可选参数可以在命令中忽略,此时将使用默认参数。可选的参数如下:
\begin{enumerate}
\item \verb|--output|:输出模型所在目录,默认为\verb|train|目录。
\item \verb|--validation|:验证集源端文件所在路径。验证最优模型存储在输出目录的\verb|eval|子目录中。
\item \verb|--references|:验证集参考译文所在路径。支持多个参考译文。
\item \verb|--parameters|:指定其他可选参数,这些参数在下一小节中描述。可选参数按照\verb|name=value|的方式给出,并且多个参数间用逗号进行分隔。具体可以参考\verb|tf.contrib.training.parse_values|的文档。
\item \verb|--help|:显示帮助信息。
\end{enumerate}


\subsection{通用参数}
\begin{enumerate}
\item \verb|num_threads|: 数据处理时使用的线程数目。
\item \verb|buffer_size|: 数据处理时的缓冲大小。
\item \verb|batch_size|: 训练时批量的大小。
\item \verb|constant_batch_size|: 是否采用固定的批量大小。如果为True,则输入数据包含\verb|batch_size|个数个句子。若为False,则输入数据中大致含有\verb|batch_size|个词。
\item \verb|max_length|: 限制训练数据中句子的最大长度。
\item \verb|train_steps|: 训练的总步数。
\item \verb|save_checkpoint_steps|: 按步设置保存频率,与\verb|save_checkpoint_secs|互斥。
\item \verb|save_checkpoint_secs|: 按秒设置保存频率,与\verb|save_checkpoint_steps|互斥。
\item \verb|initializer|: 使用不同的初始化函数。可选值为\verb|uniform_unit_scaling|,\\ \verb|uniform|,\verb|normal|以及\verb|normal_unit_scaling|。
\item \verb|initializer_gain|: 设置初始化的范围。不同的输出化函数会产生不同的作用。
\item \verb|learning_rate|: 设置初始学习率。
\item \verb|learning_rate_decay|: 设置学习率衰减函数。可选值为\verb|noam|(Transformer架构的学习率衰减方法)、\verb|piecewise_constant|(按照区间进行衰减)以及\verb|none|(不使用衰减)。
\item \verb|learning_rate_boundaries|: 参考\verb|tf.train.piecewise_constant|。设置学习率减小的边界。
\item \verb|learning_rate_values|: 参考\verb|tf.trian.piecewise_constant|。设置不同区间学习率的值。
\item \verb|keep_checkpoint_max|: 最大保留的模型存储的个数。
\item \verb|keep_top_checkpoint_max|: 保留最好模型的个数。
\item \verb|eval_steps|: 每隔\verb|eval_steps|步验证一次模型。与\verb|eval_secs|互斥。
\item \verb|eval_secs|: 每隔\verb|eval_secs|秒验证一次模型。与\verb|eval_steps|互斥。
\item \verb|eval_batch_size|: 验证时采用的句子批量大小。
\item \verb|decode_alpha|: 解码时长度惩罚,见\citep{Wu:16}中的解码公式。
\item \verb|beam_size|: Beam Search时的Beam大小。
\item \verb|decode_length|: 通过源端长度加\verb|decode_length|计算解码句子最大的长度。
\item \verb|decode_constant|: 解码时长度惩罚中的常量,见\citep{Wu:16}中的解码公式。
\item \verb|device_list|: 使用GPU的编号,支持多GPU并行计算。例如,可以使用\verb|device_list=[0,1]|选择使用第0和1号GPU。此时每个GPU会并行处理\verb|batch_size|的数据。
\end{enumerate}

\subsection{Seq2Seq参数}
\begin{enumerate}
\item \verb|rnn_cell|: 使用的RNN单元。目前仅支持LSTM。
\item \verb|embedding_size|: 词向量大小。
\item \verb|hidden_size|: RNN隐层大小。
\item \verb|num_hidden_layers|: RNN层数。
\item \verb|dropout|: Dropout大小。
\item \verb|label_smoothing|: 标签平滑大小。
\item \verb|reverse_source|: 是否将源端句子反转,见\citep{Sutskever:14}。
\item \verb|use_residual|: 多层RNN时是否使用残差连接。
\end{enumerate}
\subsection{RNNsearch参数}
\begin{enumerate}
\item \verb|rnn_cell|: 使用的RNN单元。目前仅支持GRU。
\item \verb|embedding_size|: 词向量大小。
\item \verb|hidden_size|: RNN隐层大小。
\item \verb|maxnum|: Maxout层的参数。
\item \verb|dropout|: Dropout大小。
\item \verb|label_smoothing|: 标签平滑大小。
\end{enumerate}
\subsection{Transformer参数}
\begin{enumerate}
\item \verb|hidden_size|: 网络中词向量以及每层输出的大小。
\item \verb|filter_size|: 前馈层隐层的大小。
\item \verb|num_encoder_layers|: 编码器层数。
\item \verb|num_decoder_layers|: 解码器层数。
\item \verb|num_heads|: Multi-head注意力机制中head数目。
\item \verb|shared_embedding_and_softmax_weights|: 共享Softmax层以及目标端词向量的参数。
\item \verb|shared_source_target_embedding|: 共享源端与目标端词向量参数。
\item \verb|residual_dropout|: 残差连接的Dropout大小。
\item \verb|attention_dropout|: 注意力机制中Dropout大小。
\item\verb|relu_dropout|: 前馈层中隐层Dropout大小。
\item \verb|label_smoothing|: 标签平滑大小。
\end{enumerate}


\subsection{测试}
\subsubsection{测试使用的Python脚本}
\verb|bin|目录下的\verb|translator.py|脚本用来翻译测试集中的句子。它有以下的参数可以指定:

\begin{everbatim}
usage: translator.py [<args>] [-h | --help]

Translate using existing NMT models

optional arguments:
  -h, --help            show this help message and exit
  --input INPUT         Path of input file
  --output OUTPUT       Path of output file
  --checkpoints CHECKPOINTS [CHECKPOINTS ...]
                        Path of trained models
  --vocabulary VOCABULARY VOCABULARY
                        Path of source and target vocabulary
  --models MODELS [MODELS ...]
                        Name of the model
  --parameters PARAMETERS
                        Additional hyper parameters
\end{everbatim}

用户必须指定以下的参数才可以运行\verb|translator.py|

\begin{enumerate}
\item \verb|--input|:需要进行翻译的文件。
\item \verb|--output|:输出的文件名。
\item \verb|--checkpoints|:训练好的模型所在的目录,支持多个模型集成。
\item \verb|--vocabulary|:源端与目标端的词表位置。
\item \verb|--models|:所使用的架构名称,支持不同架构之间的集成。
\end{enumerate}

\verb|translator.py|的可选参数如下:
\begin{enumerate}
\item \verb|--parameters|:调整解码中的其他参数。比如\verb|beam_size|以及GPU编号等。
\item \verb|--help|:显示帮助信息。
\end{enumerate}

\subsubsection{解码}
\verb|data|目录下的源端测试文件\verb|test.src|内容如下:
\begin{everbatim}
她 喜欢 画画 么 ?

他 一点 都 不 喜欢 画画 。

\end{everbatim}

给定一个训练好的模型目录\verb|train|,可以使用以下的命令来进行翻译:

\begin{everbatim}
python /PATH/TO/THUMT/thumt/bin/translator.py
  --models transformer
  --input test.src
  --output test.trans
  --checkpoints train
  --vocabulary vocab.zh vocab.en
  --parameters=device_list=[0]
\end{everbatim}
注意\verb|test.trans|是THUMT输出的翻译结果。


\subsection{模型平均}
模型平均也成为存储点集成(checkpoint ensemble),用于平均一次训练中存储的多个模型的参数。我们提供\verb|checkpint_averaging.py|脚本来完成这个步骤,它具有以下的参数:
\begin{everbatim}
usage: average.py [<args>] [-h | --help]

Average checkpoints

optional arguments:
  -h, --help            show this help message and exit
  --path PATH           checkpoint dir
  --checkpoints CHECKPOINTS
                        number of checkpoints to use
  --output OUTPUT       output path
\end{everbatim}

各个参数的含义如下:
\begin{enumerate}
\item \verb|path|:存储点所在的目录。通常是训练时指定的目录。
\item \verb|--checkpoints|:需要平均的存储点数目。脚本将自动读取最近的\verb|N|个模型进行平均。
\item \verb|--output|:输出的目录。在解码时指定该目录即可使用平均后的模型。
\item \verb|--help|:显示帮助信息。
\end{enumerate}


\subsection{运行例示}
以下是训练Transformer模型的命令。该模型共享源端、目标端以及softmax参数,使用4块GPU进行训练:
\begin{everbatim}
python /PATH/TO/THUMT/thumt/bin/trainer.py 
  --input train.tok.clean.bpe.32000.en.shuf 
          train.tok.clean.bpe.32000.de.shuf 
  --model transformer --output train/ 
  --vocabulary vocab.bpe.32000 vocab.bpe.32000 
  --validation newstest2013.tok.bpe.32000.en 
  --references newstest2013.tok.bpe.32000.de 
  --parameters=batch_size=6250,device_list=[0,1,2,3],
               eval_steps=5000,train_steps=100000,
               save_checkpoint_steps=1500,
               shared_embedding_and_softmax_weights=true,
               shared_source_target_embedding=true
\end{everbatim}
训练完成以后,使用以下命令进行模型平均:
\begin{everbatim}
python /PATH/TO/THUMT/thumt/scripts/checkpoint_averaging.py
  --path train --checkpoints 5 --output average
\end{everbatim}
以下是解码Transformer模型的命令:
\begin{everbatim}
python /PATH/TO/THUMT/thumt/bin/translator.py 
  --model transformer 
  --checkpoints average 
  --input newstest2014.tok.bpe.32000.en 
  --vocabulary vocab.bpe.32000 vocab.bpe.32000 
  --output output.txt 
\end{everbatim}

\bibliographystyle{apalike}
\bibliography{thumt}

\end{document}
